\documentclass[12pt,a4paper,oneside]{book}                      % tipo de documento
\usepackage[top=3cm,bottom=2cm,right=2cm,left=3cm]{geometry}    % margem
\usepackage[utf8x]{inputenc}                                    % acentuacao
\usepackage{ucs}
%\usepackage[light,math]{anttor} 								% fonte. http://www.tug.dk/FontCatalogue/
\usepackage[T1]{fontenc}
\usepackage[brazil]{babel}                                      % hifenizacao
\usepackage{amsmath}                                            % pacote matemática
\usepackage{indentfirst} 										% identacao dos paragrafos
\usepackage{hyperref} 											% hiperlinks (itens clicaveis)
\usepackage{graphicx} 											% ferramentas graficas
\usepackage{amsmath} 											% ferramentas matematicas
\usepackage{amsfonts}											% ferramentas matematicas
\usepackage{amssymb}											% ferramentas matematicas
\usepackage{pdfpages}											% inclusao de paginas em pdf
\usepackage{epstopdf}											% inclusao de figuras em eps
\usepackage{xcolor}												% habilita cores (capa)
\usepackage[some]{background} 									% papel de parede (capa)
\usepackage{color,soul} %para marcação em amarelo do texto
\usepackage[titletoc]{appendix} %para apêndices
\renewcommand\appendixtocname{Apêndices}
\usepackage{longtable} %para tabelas que ultrapassam uma página
\usepackage{framed} %para caixas ao redor do texto


% CONFIGURACAO DA CAPA
%
\definecolor{titlepagecolor}{cmyk}{0.10,.05,.05,.3} % definicao de cor

\backgroundsetup{
scale=1,
angle=0,
opacity=1,
contents={\begin{tikzpicture}[remember picture,overlay]
 \path [fill=titlepagecolor] (current page.west)rectangle (current page.north east); 
 %\draw [color=white, very thick] (5,0)--(5,0.5\paperheight);
\end{tikzpicture}}
} % configura o plano de fundo






\makeatletter                   
\def\printauthor{%                  
    {\large \@author}}          
\makeatother
% FIM DA CONFIGURACAO DA CAPA